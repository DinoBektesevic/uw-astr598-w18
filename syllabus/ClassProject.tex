\documentclass[10pt]{article}

\usepackage[letterpaper, portrait, margin=1.25in]{geometry}

\usepackage{authblk}
\usepackage[yyyymmdd,hhmmss]{datetime}
\usepackage{fancyhdr}
\pagestyle{fancy}
\lhead{ASTR 598: Astro-statistics and Machine Learning}
\rhead{Winter 2018}
\rfoot{\em \tiny Compiled on \today\ at \currenttime}
\cfoot{\thepage}
\lfoot{}


%%%%%%%%%%%%%%%%%%%%%%%%%%%%%%%%%%%%%%%%%%%%%%%%%%%%
%%% author-defined commands
\newcommand\about     {\hbox{$\sim$}}
\newcommand\x         {\hbox{$\times$}}
\newcommand\othername {\hbox{$\dots$}}
\def\eq#1{\begin{equation} #1 \end{equation}}
\def\eqarray#1{\begin{eqnarray} #1 \end{eqnarray}}
\def\eqarraylet#1{\begin{mathletters}\begin{eqnarray} #1 %
                  \end{eqnarray}\end{mathletters}}
\def\non    {\nonumber \\}
\def\DS     {\displaystyle}
\def\E#1{\hbox{$10^{#1}$}}
\def\sub#1{_{\rm #1}}
\def\case#1/#2{\hbox{$\frac{#1}{#2}$}}
\def\about  {\hbox{$\sim$}}
\def\x      {\hbox{$\times$}}
\def\ug               {\hbox{$u-g$}}
\def\gr               {\hbox{$g-r$}}
\def\ri               {\hbox{$r-i$}}
\def\iz               {\hbox{$i-z$}}
\def\a                {\hbox{$a^*$}}
\def\O                {\hbox{$O$}}
\def\E                {\hbox{$E$}}
\def\Oa               {\hbox{$O_a$}}
\def\Ea               {\hbox{$E_a$}}
\def\Jg               {\hbox{$J_g$}}
\def\Fg               {\hbox{$F_g$}}
\def\J                {\hbox{$J$}}
\def\F                {\hbox{$F$}}
\def\N                {\hbox{$N$}}
\def\dd               {\hbox{deg/day}}
\def\mic              {\hbox{$\mu{\rm m}$}}
\def\Mo{\hbox{$M_{\odot}$}}
\def\Lo{\hbox{$L_{\odot}$}}
\def\comm#1           {\tt #1}
\def\refto#1          {\ref #1}
\def\T#1              {({\bf #1})}
\def\H#1              {({\it #1})}

%%%%%%%%%%%%%%%%%%%%%%%%%%%%%%%%%%%%%%%%%%%%%%%%%%%%


\title{{\bf ASTR 598: Astro-statistics and Machine Learning}} 
\author{Andy Connolly \& \v{Z}eljko Ivezi\'{c}}
\affil{University of Washington, Winter Quarter 2018}
\date{\vspace{-5ex}}

\begin{document}
\maketitle

\vskip 0.3in
\leftline{\Large {\bf  Class Project:} Deep Proper Motion Catalog for SDSS Stripe 82 Region} 
\vskip 0.3in

{\bf Scientific Goals:}
The main scientific aim of this quarter-long class project is to produce a catalog
with improved proper motion measurements for faint stars ($14 < r < 22$) in the SDSS Stripe 
82 region (250 sq. deg.). The SDSS-based catalog\footnote{Bramich et al. 2008, ``Light and motion in SDSS 
Stripe 82: the catalogues'', MNRAS 386, 887.} for 3.7 million objects
constructed by Bramich et al. will be augmented with newer additional data obtained
with the Dark Energy Camera\footnote{See, for example, http://legacysurvey.org/decamls/}, 
and possibly from other surveys, and the proper motions will be refit. If all goes well,
the results should be publishable in a top astronomical journal. 

\vskip 0.2in
{\bf Learning Goals:}
While working on this project, students will develop a working knowledge of
the NOAO Datalab interface, astropy, pandas, astroML and other astronomical
python tools, and of selected methods from astro-statistics (e.g., robust
regression, Bayesian statistics, clustering, visualization). 

\vskip 0.2in
{\bf Prerequisites:}
The students taking this class are required to open an account at the NOAO Data Lab 
site\footnote{http://datalab.noao.edu}. The Data Lab allows users to
\begin{itemize}
\item access, search, and filter databases containing large catalogs;
\item create custom databases and analyses from large catalogs using familiar tools;
\item combine catalog databases with data from NOAO telescopes, analysis results, and data from external archives in one place; 
\item share custom results easily with collaborators and create and publish catalogs derived from large data sets through a central workspace;
\item experiment with tools being developed for LSST using existing large data sets.
\end{itemize}

We will use GitHub and Jupyter notebooks for progress tracking. 

\vskip 0.2in
{\bf Brief Project Outline}

We will discuss detailed work plan in class and here is only an overview of the
main steps: 
\begin{enumerate}
\item Download Bramich et al. catalog (HLC files) from the SDSS
website\footnote{http://das.sdss.org/va/stripe\_82\_variability/SDSS\_82\_public/}
and perform preliminary analysis (e.g., the variation of proper motion errors with 
 position and magnitude; also, see examples in the last bullet below and, in particular,
    reproduce Figs. 22 and 23 from Sesar et al.); Using Data Lab, perform similar analysis
     for DECaLS and other available data from the Stripe 82 region.
\item Cross-match to DECaLS data using Data Lab (technical details TBD). 
\item Preliminary analysis and quality assurance of the assembled data set (e.g., 
          coordinates vs. time plots for stars with large SDSS proper motions); 
          decision whether astrometric recalibration is required (perhaps using galaxies). 
\item Fit proper motions (see eqs.3-6 in Bramich et al.)  using the Bramich et al. SDSS values as priors. 
\item Analyze updated proper motions (e.g., Fig. 2 in Vidrih et al. 2007, MNRAS 382, 515;
   Figs. 22 and 23 from Sesar et al. 2010, AJ 708, 717, color-coded with the mean proper 
   motions), including using quasars for the assessment of systematic errors. 
\item Write paper(s) and become rich and famous! 
\end{enumerate}

\end{document}






The Legacy Surveys (http://legacysurvey.org) 

The Legacy Surveys are producing an inference model catalog of the sky from a set of 
optical and infrared imaging data, comprising 14,000 deg² of extragalactic sky visible 
from the northern hemisphere in three optical bands (g,r,z) and four infrared bands. 
The sky coverage is approximately bounded by -18° < δ < +84° in celestial coordinates 
and |b| > 18° in Galactic coordinates. 

To achieve this goal, the Legacy Surveys are conducting 3 imaging projects on different 
telescopes, described in more depth at the following links:

The Beijing-Arizona Sky Survey (BASS)	
The DECam Legacy Survey (DECaLS)	
The Mayall z-band Legacy Survey (MzLS)


Current Release: Data Release 5, October 2017

Images from DECaLS g,r,z-band observations are included from August 2014 through 
May 2017. DR5 also includes DECam data from a range of non-DECaLS surveys, including 
observations that were conducted from September 2012 to May 2017.

DR5 imaging is first reduced through the NOAO Community Pipeline before being processed 
using the Tractor. The optical data covers a disjoint footprint with, roughly, 7400 deg² having 
at least one observation in g-band, 8000 deg² having at least one observation in r-band and 
10200 deg² having at least one observation in z-band. A total of 6800 deg² is covered by at least 
one observation in all three optical filters.

There are approximately 680 million unique sources in DR5 (371 million are unresolved). In DR5, five 
morphological types are used: point sources, round exponential galaxies with a variable radius ("REX"), 
deVaucouleurs profiles (elliptical galaxies), exponential profiles (spiral galaxies), and composite profiles 
that are deVaucouleurs + exponential (with the same source center). 

Co-added images and Tractor catalogs are presented in "bricks" of approximate size 0.25° × 0.25°. Each brick 
is defined in terms of a box in RA,Dec coordinates. For the image stacks, we use a simple tangent-plane (WCS TAN) 
projection around the brick center. The projections for the g,r,z filters are identical.

The flux calibration for DECaLS is on the AB natural system of the DECam instrument. The natural system means 
that we have not applied color terms to any of the photometry, but report fluxes as observed in the DECam filters.

Zero point magnitudes for the CP version 2 reductions of the DECam images were computed by comparing 
7″ diameter aperture photometry to PS1 photometry, where the latter was modified with color terms to place 
the PS1 photometry on the DECam system. The same color terms are applied to all CCDs. Zero points are computed 
separately for each CCD, but not for each amplifier. The color terms to convert from PS1 to DECam were computed for stars in the color range 0.4<(g−i)<2.7
 as follows:...


The brightnesses of objects are all stored as linear fluxes in units of nanomaggies (3.631e-29 erg/cm2/Hz).

The median 5σ point source (AB) depths for areas with 3 observations in DECaLS was g=24.65, r=23.61, z=22.84. 




http://adsabs.harvard.edu/abs/2017MNRAS.470.1259D

Deason et al. (2017) combined Gaia DR1 with SDSS and obtained proper motion errors of 2 mas/yr 
down to r=20. With this dataset, they measured net prograde rotation of halo stars (RR Lyrae, 
BHB and K giant samples) of 14 ± 2 ± 10 (syst.) km/s, out to Galactocentric radius of 50 kpc.

Data compilation for halo streams: Grillmair and Carlin (2016) 
https://arxiv.org/pdf/1603.08936.pdf